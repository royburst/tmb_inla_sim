  

\title{Fitting geostatistical models in INLA and TMB: implementation of under-5 mortality mapping in both packages.}


\date{\today}

\documentclass[12pt]{article}

\usepackage{outlines}
\renewcommand{\theenumi}{\Roman{enumi}. }
\renewcommand{\labelenumi}{\theenumi}
\renewcommand{\theenumii}{\Alph{enumii}. }
\renewcommand{\labelenumii}{\theenumii}
\renewcommand{\theenumiii}{\roman{enumiii}. }
\renewcommand{\labelenumiii}{\theenumiii}
\renewcommand{\theenumiv}{\alph{enumiv}) }
\renewcommand{\labelenumiv}{\theenumiv}


\begin{document}
\maketitle
%%%%%%%%%%%%%%%%%%%%%%%%%%%%%%%%%%%%%%%%%%%%%%%%%%%%%%%%%%%%%%

\begin{outline}[enumerate]
   \1 Aims of the project
      \2 To understand and be able to describe the models and how INLA and TMB fit these models, their relative benefits and drawbacks.
        \3 A written component
        \3 A simulation experiment with runtime tests. 
      \2 To replicate the work already done in INLA in TMB.
      \2 To set ourselves up to extend the TMB model to deal with both point and polygon data.
      
      
      
    \1 Background and context
      \2 Intro, discuss importance of sub-national estimates.
      \2 Target is to produce a high resolution spatial map of under-5 mortality across Africa. Making predictions at places were we do not have data requires statistical inference - natural choice is model based geostatistics.
      \2 Describe the data. Binomial outcomes, estimating the probability of events (deaths) over exposures (months lived). For the moment we are dealing exclusively with point data - either as geo-referenced complete birth histories or polygon data that is abstracted down to weighted point data. 
      
    \1 Structure of spatio-temporal processes to be modelled
      \2 Describe the GLMM and the spatio-temporal random effect for binomial data with a logit link.
      \2 Equations and explanations, overview of MBG models in general and this one specifically. SPDE and all that good stuff.
      
    \1 Estimation with INLA: How the model is fitted in INLA
      \2 Describe, in words, what it is doing
      \2 Code examples
      
    \1 Estimation with TMB: How the model is fitted in INLA
      \2 Describe, in words, what it is doing
      \2 Code examples
            

    \1 Results of simulation experiments.
      \2 Show we are able to produce the same results in both implementations
      \2 Compare computational time with both and plot how these scale with growing data sets
      
      
    \1 Discussion of future directions of this work, and what secific things we can do with TMB going forward to improve our work.
   
   
   
   
   
   
   
   
   
   
   
   
   
   
   
   
   
   
   
\end{outline}


\end{document}
